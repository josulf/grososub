\tocchapter{Azpititulu formatuak}

Azpititulu formatu ugari daude gaur egun, baina erabilenak \textit{Subrip}(SRT) eta \textit{Advanced Substation Alpha}(ASS) formatuak dira. Bi formatu hauek testu fitxategiak dira, beraz, edozein testu editorerekin ireki eta eraldatu ditzakegu, baina lan handia suposatzen du. Horregatik hainbat tresna sortu dira hauek modu eroso batean manipulatzeko.

\section{SRT}
SRT formatua oso simplea da, hona hemen adibide bat:
\begin{verbatim}
1
00:00:20,000 --> 00:00:24,400
In connection with a dramatic increase
in crime in certain neighbourhoods,

2
00:00:24,600 --> 00:00:27,800
The government is implementing a new policy...
\end{verbatim}

Hasieran identifikadorea daukagu eta ondoren azpititulu hori pantailan noiz agertuko eta noiz desagertuko den. Azkenik, zein den erakutsi behar den testua eta amaitzeko lerro txuri bat.

\section{ASS}
ASS formatua (SSA formatuaren eboluzioa) askoz konplexuagoa da, aukera asko eskeintzen dituelako:

\begin{itemize}
\item Pertsonaia desberdinentzako \textit{estiloak} erabil ditzakegu, letra mota, tamaina, koloreak, itzala, angelua, transparentzia, etab. definituz
\item Azpititulu bakoitzaren posizioa eta itxura alda dezakegu komando batzuen bidez.
\item Azpititulu desberdinen arteko \textit{talkak} modu desberdinetan trata ditzakegu.
\item Azpitituluak denboran zehar alda ditzakegu, pantailan zehar mugituz, koloreak aldatuz, etab.
\item Karaoke sinpleak sortu ditzakegu abestietarako.
\item \textit{Typesetting}-erako tresna ugari ditugu, irudian agertzen diren kartelak ezabatzeko eta hauen gainean idazteko adibidez.
\end{itemize}

\subsection{Barne egitura}
ASS formatuak zenbait azpiatal dauzka bere barruan, eta guztiak garatuko den programak maneiatu behar ditu.

\subsubsection{Goiburukoak}
Fitxategi hasieran zenbait goiburuko agertzen dira, erabilienak eta estandarrean daudenak hauek izanik:
\begin{itemize}
	\item Izena.
	\item Jatorrizko izena.
	\item Script mota.
	\item Atal desberdinen egileak.
\end{itemize}
Hauetatik garrantzitsuena script mota da, eta ASS formatuan beti \textbf{v4.00+} izango da. Balio baxuago bat badu SSA formatu zaharrean egongo dira eta gure aplikazioak ez ditu onartuko, egia esanda formatu hau duela zenbait urte ez delako erabiltzen.

Dena den, goiburuko gehiago ager daitezke, adibidez oso tipikoa da fitxategia sortzeko erabili den programa eta honen bertsio gordetzea, edo programaren funtzionamendurako interesgarriak diren datuak gordetzea, adibidez zein den fitxategi horrekin ireki den azkenengo audio edo bideo fitxategia.

Atal honen hasieran \texttt{[Script Info]} lerroa aurkituko dugu, eta informazioa \texttt{Gakoa: Balioa} moduan egongo da. Adibidez hauek dira gure programak sortzen dituen defektuzko goiburukoak (";" karaktereaz hasten diren lerroak komentarioak dira eta ez dira kontuan hartuko):
\begin{verbatimtab}[8]
[Script Info]
; Script generated by GrosoSub v0.01
Title: Default GrosoSub script
ScriptType: v4.00+
PlayResX: 640
PlayResY: 480
\end{verbatimtab}

\subsubsection{Estiloak}
Hurrengo atalean fitxategian definituta dauden estiloak agertuko dira, eta \texttt{[V4+ Styles]} lerroarekin hasiko da atal hau. Hurrengo lerroan estiloak definitzeko formatua aurkituko dugu, baina lerro hau baztertu dezakegu ASS formatuan beti berdina izango delako. Ondoren lerro bakoitzean estilo bat definituta egongo da, eta minimoki bat egongo da definituta, \textit{Default} izena duena. Estilo bakoitzean honakoak definitzen dira:
\begin{itemize}
	\item Izena.
	\item Tipografia.
	\item Tipografia tamaina.
	\item 4 Kolore (lehenengoa, bigarrengoa, bordearena eta itzalarena).
	\item Belztuta.
	\item Kurtsiba.
	\item Azpimarratuta.
	\item Gainmarratuta.
	\item X eta Y ardatzetako luzapena.
	\item Karaktereen arteko espazioazioa.
	\item Karaktereen angulua.
	\item Bordearen mota.
	\item Bordearen tamaina.
	\item Itzalaren tamaina.
	\item Pantailako posizioa.
	\item Ezkerreko, eskuineko eta azpiko marginak.
	\item Karaktere kodifikazioa.
\end{itemize}
Aipatu beharra dago ASS formatua ez dagoela guztiz ondo definituta. Definizio erabiliena \textit{VSFilter} Windows-erako programak inplementatzen duena da, ASS motako azpitituluak bideoaren gainean erakusteko balio duen programa batena hain zuzen ere. Adibidez inplementazioa batzutan itzalaren tamaina osokoa izan behar da, baina beste batzuetan zenbaki erreala da. Kuriotsitate moduan, balio boolearrak adieratzeko -1 erabiltzen da \texttt{true} adierazteko eta 0 \texttt{false} adierazteok. Bordearen tamainak onartzen dituen balioak 0 (borde normala) edo 3 (kutxa bat inguruan) dira.

Hona hemen estiloen ataleko adibide bat:
\begin{verbatimtab}[8]
[V4+ Styles]
Format: Name, Fontname, Fontsize, PrimaryColour,
	 SecondaryColour, OutlineColour, BackColour,
 	 Bold, Italic, Underline, StrikeOut, ScaleX,
	 ScaleY, Spacing, Angle, BorderStyle, Outline,
	 Shadow, Alignment, MarginL, MarginR,
	 MarginV,Encoding
Style: Default,Arial,20,&H00FFFFFF,&H0000FFFF,
	 &H00000000,&H00000000,0,0,0,0,100,100,
	 0,0,1,2,2,2,10,10,10,0
\end{verbatimtab}

\subsubsection{Gertakariak}
Azken atal honetan lerro bakoitzean azpititulu bat agertuko da informazio gehigarriarekin. Kasu honetan atala \texttt{[Events]} lerroarekin hasiko da, eta aurrekoan bezala hurrengo lerroa baztertu dezakegun lerro bat da, hurrengo lerroen formatua adierazten duena. Hurrengo lerro bakoitzean honako informazioa aurkitu dezakegu:
\begin{itemize}
	\item Dialogo bat edo komentario bat den.
	\item Azpititulua zein geruzakoa den.
	\item Hasiera eta bukaera denborak.
	\item Zein estilorekin agertu behar den pantailan.
	\item Zein pertsonaiak esaten duen.
	\item Ezkerreko, eskuineko eta beheko marginak.
	\item Efektua.
	\item Testua.
\end{itemize}
Dialogoa bada, pantailan agertuko da, eta komentarioa bada bideoarekin batera ikusterakoan lerro hori ez da agertuko, iruzkinak jartzeko erabiltzen da. Bi azpititulu edo gehiago une konkretu batean agertzen badira eta geruza berdinean badaude, \textit{kolisio} bat egon dela esango da, eta bata bestearen gainean agertuko da (goiburukoetan kolisioen konportamendua konfiguratu daiteke). Adierazten den estiloa script-aren estilo atalean egon behar da, eta pertsonaia modu informatiboan erabiliko da soilik. Efektua gaur egun ez da ezertarako erabiltzen, eta azkenik Testua pantailan agertuko den karaktere sekuentzia izango da. Karaketere sekuentzia honetan ASS komandoak sartu daitezke, adibidez koloreak aldatzeko (tarte batean soilik), testua mugitzeko, etab. Komando hauek kortxete artean agertuko dira, eta pantailan erakustean komando hauek interpretatu egingo dira.

Hona hemen atal honen adibide bat:
\begin{verbatimtab}[8]
[Events]
Format:  Layer, Start, End, Style, Name, MarginL,
	  MarginR, MarginV, Effect, Text
Dialogue: 0,0:01:21.86,0:01:23.76,Default,BLA,0000,
		0000,0000,,Ah, es verdad.
Dialogue: 0,0:01:30.37,0:01:33.51,Default,BLA,0000,
		0000,0000,,Ya no estoy más en aquel mundo.
Dialogue: 0,0:00:00.00,0:00:00.00,Default,CARTEL,0000,
		0000,0000,,Hoshi no Koe. Voces de una 
			   Estrella Lejana.
Dialogue: 0,0:01:39.21,0:01:40.60,Default,BLA,0000,
		0000,0000,,¡Espera, Noboru-kun!
Dialogue: 0,0:01:41.75,0:01:42.49,Default,BLA,0000,
		0000,0000,,Nagamine...
\end{verbatimtab}

\subsubsection{Informazio gehiago}
Fitxategien formatuari buruzko informazio gehigarria \textit{Matroska}-ren web orrian\cite{ma:ass} aurkitu dezakegu, eta ASS komandoen sintaxiari buruzko informazioa SourceForge-en dagoen \textit{guliverkli} proiektuan\cite{gu:ass}.
