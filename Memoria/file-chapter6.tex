\tocchapter{Ondorioak}
Proiektu honen ondorioak aztertzeko orduan, bi alderdi hartu behar dira kontuan: alde batetik garapena eta honek ekarritako emaitzak kontuan hartuta eta bestetik planifikazio eta gestioa kontuta hartuta.

Garapenaren aldetik, hasieran zehaztu ziren helburuak lortu dira eta amaieran lortu den aplikazioa guztiz funtzionala da. Garapenean zehar zenbait ezagutza bereganatu dira, adibidez Cocoa API-arekin aplikazio grafikoak egiten ikasi da, baita Objective-C lengoarekin lan egitea ere. Komentatu beharra dago bi teknologia hauek gero eta gehiago erabiltzen ari direla, Mac OS X sistemak gero eta erabiltzaile gehiago dituelako azken urte hauetan. Honez gain \textit{iPhone} telefono mugikorrean programatzen diren eta gaur egun \textit{modan} dauden aplikazioak Objective-C-z eta \textit{Cocoa Touch}-ez daude idatzita (azken hau Cocoa-ren oso antzekoa da, batez ere programatzeko orduan), beraz oso interesgarria izan daiteke hauek ikastea etorkizunari begira.

Planifikazioa eta gestioari begira, proiektua ez da oso ondo joan, nahiz eta azkenen planifikatu ziren helburuak lortu. Zenbait aldiz atzerapenak egon dira, eta hauekin batera birplanifikazioak egon dira. Hasieran proiektua 2009. urteko irailan aurkeztuko zen, ikasleak pasa den kurtsoan eduki zuen lan zamagatik 2010. urteko otsailera atzeratu zen. Atzerapen berri honekin problema gehiago etorri ziren, eta proiektua berriro atzeratu zen, nahiz eta azkenean data horretarako bukatuta egon.

Dena den, ondorio bakar bat atera behar bada egin den lan osotik, ondorio positiboa da hori, nahiz eta dena zenbait aldiz birplanifikatu behar izan, bizitza errealean horrelakoak gertatuko dira nahi edo ez.
