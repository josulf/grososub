\tocchapter{Lehenengo kapitulua}

Hau lehenengo kapitulua da.

\section{Lehenengo sekzioa}

Hau lehenengo kapituluaren lehenengo sekzioa da.

Parrafoak indentatuta hasten direla ikusiko duzu, adibidez parrafo hau hala agertzen da, ikus dezakezun bezela.

\noindent Indentazioa ekidin daiteke, \verb#\noindent# jarrita. Adibidez, lerro hau.

\noindent Item zerrenda adibidea:

\begin{itemize}
 \item Hau adibidezko zita bat da\cite{ko:04} (\textit{file-bibliography.bib} bibliografia fitxerotik).
 \item Hau beste zita bat da\cite{hi:08} (\textit{file-bibliography2.bib} fitxerotik).
 \item Ikus \ref{tab}~Taula.
 \item Ikus \ref{fig}~Figura.
 \item Ikus \ref{eq}~Ek.
\end{itemize}

\begin{table}[hbt]
 \begin{center}
 \begin{tabular}{cc}
  a & b \\
  c & d 
 \end{tabular}
 \end{center}
 \caption{Hau taula bat da.}
 \label{tab}
\end{table}

\begin{figure}[hbt]
 \begin{center}
  % Hurrengo lerroa deskomentatu, FILENAME nahi den (E)PS fitxeroaz aldatuz
  %\includegraphics[width=1.0\cw]{FILENAME.eps}
 \end{center}
 \caption{Figura hau hutsik dago.}
 \label{fig}
\end{figure}

\begin{equation}
  7 = 1
 \label{eq}
\end{equation}
