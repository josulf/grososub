%
% Erabilgarriak suerta dakizkizuken paketeak (gom = gomendatua, opz = opzionala)
%
\usepackage{fancyhdr}          % (gom)  goiburu eta oinak moldatzea ahalbidetzen du.
\usepackage{courier}           % (opz)  fuente hau erabili, defektuz.
\usepackage{setspace}          % (opz)  lerroen arteko espazioa aldatzea ahalbidetzen du.
\usepackage{longtable}         % (opz)  taulak orri bat baino gehiago luzatzea ahalbidetzen du.
\usepackage{lscape}            % (opz)  \landscape komanduaz zerbait apaisatua jartzea ahalbidetzen du.
\usepackage{color}             % (opz)  koloreri loturiko zenbait komandu (halaber \color).
\usepackage{rotating}          % (opz)  PSak eta EPSak biratzeko.
\usepackage{textcomp}          % (opz)  euro simboloa gehitzea ahalbidetzen du, \texteuro simboloaz.
\usepackage{minitoc}           % (opz)  kapitulu bakoitzean ToCak (gaien indizeak) gehitzea ahalbidetzen du.
\usepackage{epsf}              % (opz)  EPSen zenbait moldaketa ahalbidetzen du.
\usepackage[utf8x]{inputenc}   % (opz)  zenbait textu editorek karaktere gehiago (adbz. tildeak) erakustea ahalbidetzen du.
\usepackage[absolute]{textpos} % (gom)  textua nahi bezala orrian kokatzeko (portadarak beharrezkoa)
\usepackage{srcltx}            % (opz)  .dvi-tik .tex-era pasatzea ahalbidetzen du.
\usepackage[basque]{babel}     % (gom)  LaTeX euskarara moldatzeko.

%
% Marginak moldatzeko. Deskomentatu eta aldatu egiten ari zaren baldin badakizu.
% Ohar ematen diren baloreei pulgada bat gehitzen zaiola. Emandako baloreak A4
% papera eta itsas_pfc.cls estilorako aurredeterminatutakoak dira.
%
%\setlength{\oddsidemargin}{10pt}     % ezker margina orri ezpareetarako (ezkerra)
%\setlength{\evensidemargin}{52pt}    % ezker margina orri pareetarako (eskubia)
%\setlength{\textwidth}{390pt}        % textu gorputzaren zabalera

%
% Figuren kokapen automatikoa hobetzeko.
% (http://dcwww.camp.dtu.dk/~schiotz/comp/LatexTips/LatexTips.html#captfont-tik hartua)
%
\renewcommand{\topfraction}{0.85}
\renewcommand{\textfraction}{0.1}
\renewcommand{\floatpagefraction}{0.75}

%
% Orriaren goikaldea eta textua hasten deneko puntuaren arteko distantzia (hor doaz
% goiburuak). LaTeX kexatu egiten da 15pt baino txikiagoa denean.
%
\headheight 15pt

%
% Portadan erabiltzen den textpos paketearentzat.
%
\setlength{\TPHorizModule}{\paperwidth}
\setlength{\TPVertModule}{\paperheight}
\newcommand{\tb}[4]{\begin{textblock}{#1}[0.5,0.5](#2,#3)\begin{center}#4\end{center}\end{textblock}}

%
% Hemen zure komandoak defini ditzakezu.
% 
% \newcommand{cmd}[args]{def}
%
% cmd  = definitu nahi den komandoaren izena (adbz. \ura)
% args = argumentu kopurua
% def  = definizioa, non #1, #2... lehen, bigarren... argumentuekin trukatuko diren.
%
% Adibidez:
%
% \newcommand{\ura}[1]{H\ensuremath{_#1}O}
%
% "\ura{33}" idaztean, output-ean H33O aterako da (33 azpiindezea delarik).
%

%\newcommand{\zerbait}{zerbait}

%
% Hemen LaTeX-i esan diezaiokezu nondik moztu berak mozten ez dakizkien hitzak.
% Adibidez, "gnomonly" nola gidoiak (-) dauden lekutik mozteko:
%
\hyphenation{gno-mon-ly} 
 
%
% Hasieran orriak zenbaki erromatarrez zenba daitezela.
% Aurrerako zenbaki arabiarretara pasako gara.
%
\pagenumbering{Roman}
