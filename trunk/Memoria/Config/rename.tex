%
% Fitxero honek LaTeX-en izen (aldagai) interno batzuk ditu, eta
% beraien balioa alda dezakezu. Adibidez, kapituluak "Sekzio" dei
% daitezela egin dezakezu.
%
\renewcommand\bibname{Bibliografia}                 % Bibliografia sekzioaren izena hori izan dadila.
\newcommand{\myname}{Josu López Fernández}                     % autorearen izena
\newcommand{\myboss}{Juan Antonio Pereira Varela}                     % gainbegiralearen izena
\newcommand{\thesistitle}{ASS formatuko azpititulu editorea Mac OS X sistemarako}             % lanaren izena
\newcommand{\worktype}{Karrera Bukaerako Proiektua} % lan mota
\newcommand{\logo}{Utils/ehu_logo.eps}              % logoa duen fitxeroa (adbz. portadarako).
%\renewcommand{\tablename}{xxx}                     % taula azpiko izena (xxx 1: bla-bla-bla)
%\renewcommand{\figurename}{xxx}                    % figura azpiko izena (xxx 1: bla-bla-bla)
%\renewcommand{\listtablename}{yyy}                 % taulen indizearen izena.
%\renewcommand{\listfigurename}{yyy}                % figura indizearen izena.
