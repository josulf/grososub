\tocchapter{Sarrera}
\section{Azpitituluak}
Telebistarako eta zinemarako sortzen den material gehiena bikoiztu egiten da gure hizkuntzara, hemen ez dagoelako ohitura atzerrian sortzen diren produktuak jatorrizko hizkuntzan ikusteko. Hizkuntza batetik bestera pasatzerakoan moldaketa prozesu bat dago, guk entzuten duguna aktorearen ahoarekin sinkronizatuta egon behar delako. Honen ondorioz, zenbait gauza \textit{galdu} egiten dira bikoizketa prozesuan zehar.

Azpitituluak erabiliz ordea, arazo hau ekidin daiteke, pantailan agertzen zaigun testua eta aktorearen ahoa ez direlako guztiz sinkronizatuta egon behar. Dena den, azpitituluek zenbait arau bete behar dituzte: oso luzeak ez izatea, ikusgarriak izatea, denbora ematea irakurtzeko, etab.

Azken urteetan, Internet-en bultzadarekin gehienbat, talde altruista asko sortu dira azpitituluak egiten dituztenak, jatorrizko hizkuntzan materiala ikusi ahal izateko. Bestalde, honen bidez, itzulita iritsiko ez den materiala ikus dezakegu gure hizkuntzan azpititulatuta.

\section{Azpitituluak nola ikusi}
Bi aukera dauzkagu azpitituluak ikusteko bideo baten gainean:

\textit{Hardsub}: bideoa kodetzerakoan bideoan bertan txertatu. Azpitituluak bideoaren parte dira, beraz, ez dugu software berezirik behar hauek ikusteko. Desabantaila bezala, bideoan kodetuta daudenez ezin ditugu atera editatzeko edo erreproduzitzerakoan ezkutatu, eta zenbait sistema eragileetarako ez daude bideoan txertatzeko plugin-ak edo aplikazioak.

\textit{Softsub}: erreproduzitzerako orduan bideo gainean ezarri. Horretarako bideo fitxategiaz gain azpitituluen fitxategia behar dugu (edo fitxategi berdinean sartu, \textit{Matroska}\footnote{\url{www.matroska.org}} formatuak aukera hau ematen du adibidez). Desabantaila batzuk dauzka honek: zenbait erreproduktorek ez dituzte onartzen (gehienbat ASS formatuan daudenak) eta CPU-denbora asko jan dezakete \textit{efektu} asko erabiltzen badira. Dena den, sistema eragile erabilienetan (Windows, Linux eta Mac OS X) badago erreproduktoreren bat bateragarria dena.

\section{Arazoa}
Mac OS X plataforman azpitituluak ikusteko ez dugu arazorik, VLC\footnote{\url{http://www.videolan.org/vlc/}}, mplayer\footnote{\url{http://www.mplayerhq.hu/}} edo Quicktime\footnote{\url{http://www.apple.com/quicktime/}} (Perian\footnote{\url{http://www.perian.org}} izeneko plugin batekin) erabili ahal ditugulako \textit{softsub} motako azpitituluak ikusteko (lehen esandakoarekin argi geratu da \textit{hardsub} motakoekin ez dugula arazorik izango, bideoaren parte direlako), baina sortzeko edo editatzeko arazoak dauzkagu, ez dagoelako horretarako tresna espezializaturik. Datu txikiren bat aldatzeko testu editoreren bat erabili dezakegu, baina gauza konplexuak egiteko ez, lan ikaragarria egin behar delako adibidez lerro berri bat sartzeko: bideoan edo audioan begiratu noiz hasten den eta noiz bukatzen den dialogo-lerroa, fitxategian jarri, testua itzuli, etab. Gainera fitxategien formatua dela eta oso erraza da komaren (``,'') bat ahaztea eta dena hankaz gora uztea.

Beraz, \textit{Advanced Substation Alpha} motako azpitituluak sortzeko eta editatzeko software bat behar dugu, erabiltzaileak prozesu hau modu eroso eta grafikoan egin ahal izateko.
