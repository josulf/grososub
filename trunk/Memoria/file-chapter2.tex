\tocchapter{Aurrekarien analisia}

\section{Sarrera}
Gaur egun badaude ASS formatuko azpitituluak editatzeko aplikazioak, gehienak Microsoft-en Windows sistema eragilerako eta batzuk Linux-erako. Mac OS X sistemarako ordea, ez dago erabilgarria den aplikaziorik.
\section{Azpititulu editoreak}
Atal honetan denboran zehar erabiliak izan diren azpititulu editoreak aztertu eta konparatuko ditugu.
\subsection{Sub Station Alpha}
Programa honek hasi zuen gure proiektuaren inguruko dena, programarekin batera, SSA formatua sortu zelako (honetatik atera zen gaur erabiltzen den ASS formatua). Microsoft Windows-erako doainiko programa pribatiboa da, 1996. urtean sortua britaniar baten eskutik, \textit{Kotus} ezizenarekin ezagutua. Ateratako azken bertsioa 2000. urtekoa da, beraz honen garapena duela asko bertan behera utzi zen.

Garatzailearen esanetan, programaren funtzioa azpitituluak modu erraz batean eta kalitate handiarekin sortzea zen, garaiko garestiak eta motelak ziren tresna komertzialen ordezko bat sortzea hain zuzen ere.

Nahiz eta bere garapena gaur egun kantzelatua egon, oraindik jende askok erabiltzen du, programa sinplea delako eta beraz oso erreza erabiltzeko, baita bere garaian erabiltzen zutenen ohituragatik.

Hiru fitxategirekin lan egiten du:
\begin{itemize}
\item SSA script-a.
\item Bideo fitxategi bat.
\item Audio fitxategi bat.
\end{itemize}

Azpitituluak editatzeko, SSA fitxategia da behar dugun bakarra, baina azpitituluak sinkronizatzeko audio fitxategia kargatu behar dugu. Fitxategi honen formatua \textit{PCM WAV} motakoa izan behar da, 8 bitekoa eta kanal bakarrekoa (\textit{mono}). Bideoaren erabilera erreferentzia bat edukitzeko mugatzen da, azpitituluak ez dituelako erakusten prozesatu ondoren ikusiko diren bezala.

% TODO: Kapturak eta hauen inguruko azalpenak

\subsection{Medusa}
2002. urtean, \textit{Sub Station Alpha}-ren garapena 2 urtez geldirik egon ondoren, editore hau kaleratu zuen programatzaile italiar batek, \textit{Kaidousama} ezizenarekin ezagutua. Urte honetarako, SSA formatuaren eboluzioa atera zen, \textit{Advanced Substation Alpha} (ASS), gaur egun erabiltzen dena. Aurreko programa bezala, hau baita Microsoft-en Windows sistema eragilerako garatu zen Visual Basic 6 erabiliz.

Aldaketa handiak suposatu zituen editore honek, gehienbat ASS formatuaren inplementazioagatik. Hauek dira aipagarrienak:
\begin{itemize}
\item ASS formatuaren erabilera natiboa.
\item Audio formatu gehiago erabili ditzake, adibidez \textit{MP3} edo \textit{OGG Vorbis}.
\item Bideo bidezko sinkronizazioa.
\item Sintaxiaren ¿resaltado? denbora errealean hau hobeto ulertzeko.
\item Karaokeak silaba bidez sinkonizatzeko aukera, azkarra eta erosoa.
\end{itemize}

Hala ere, lan egiteko modua bere aintzindariaren antzekoa da, horregatik lortu zuen \textit{Sub Station Alpha} erabiltzen zuten askok hau erabiltzea.

% TODO: Kapturak eta hauen inguruko azalpenak

2004. urtean bere garapena bertan behera utzi zen, egileak Delphi-n birprogramatzea erabaki zuelako (\textit{ChronoSub} izenarekin), baina ez da berririk egon honen inguruan.
\subsection{Aegisub}
Hau da gaur egungo ASS formatuko azpititulu editorerik hoberena. Azpititulatzaile komunitateak berak garatu du, \textit{Sub Station Alpha}-k eta \textit{Medusa}-k zituzten arazoak konpontzeko asmoarekin.

Hasieran \textit{Visual SSA} izenarekin ezagutu zen, ondoren \textit{Visual ASS} bezala eta azkenik \textit{Aegisub} izenarekin geratu da. 2005. urtean agertu zen v1.00 \textit{beta} bertsioa, eta 1.07 bertsiotik aurrera bere kodea askatu zuten BSD\footnote{\url{http://en.wikipedia.org/wiki/BSD_license}} lizentzia batekin. Bere denbora librea azpititulatzen pasatzen zuten bi programatzailek hasi zuten proiektua, baina gaur egun jende asko dago garapen taldean, horregatik gaur egun dagoen aplikaziorik erabilena eta konpletuena da.

C++ lengoiaz garatuta dago, eta nahiz eta hasieran Windows-erako bakarrik izan, gaur egun Linux-en funtzionatzen du, baita Mac OS X-en ere (garatzaileen esanetan, baina honen inguruan aurrerago hitz egingo dugu).

Hauek dira programak dituen ezaugarri nagusiak:
\begin{itemize}
\item ASS formatuaren erabilera natiboa, honez gain SSA, SRT eta TXT formatuak inportatu ditzake.
\item Unicode testu kodifikazioan dauden fitxategiak erabil ditzake (oso erabilgarria beste hizkuntzetako karaktereak sartzeko).
\item Karaokeak modu automatizatuan egiteko aukera \textit{Lua} programazio lengoaia erabiliz.
\item Sintaxiaren ¿resaltado?.
\item Bideoak ireki ditzake \textit{AviSynth} erabiliz.
\item Azpitituluak erakusten ditu \textit{VSFilter} erabiliz, beraz prozesatu ondoren ikusiko diren bezala.
\item \textit{DirectShow}-k dekodifikatu dezakeen edozein audio fitxategirekin lan egiteko ahalmena.
\item Karaokeen sinkronizazioarako sistema aurreratua.
\item Itzulpen prozesurako laguntzaile bat.
\item Estiloak gorde daitezke script desberdinetan erabili ahal izateko.
\item Estiloak aplikatzeko laguntzailea bat.
\item Azpitituluen arteko kolisioak antzematen ditu.
\item Makroen erabilera zenbait ataza egiteko.
\item Azpitituluak modu grafikoan editatzeko aukera.
\end{itemize}

Ikusten dugunez, funtzionalitate asko ditu programa honek, bere atzean dagoen garapen taldeari esker.

% TODO: Kapturak eta hauen inguruko azalpenak

\section{Konparaketa}
\ref{konparaketa}~taulan daude hiru programa hauen ezaugarriak. Bertan ikus dezakegunez, Aegisub da hiruretatik hoberena.
\begin{longtable}{|l|c|c|c|}
\hline
& \grey Sub Station Alpha & \grey Medusa & \grey Aegisub\\
\hline
\endhead
\hline
\caption{\label{konparaketa}Aztertutako programen ezaugarriak}
\endfoot
\grey Bertsioa & 4.08 & 0.1.2.0 & 2.00alpha\\
\hline
\grey Garapena & \red Bertan behera & \red Bertan behera & \green Aktiboa\\
\hline
\grey Lizentzia & \red Itxia & \red Itxia & \green BSD\\
\hline
\grey Lengoaia & Visual Basic & Visual Basic & C++\\
\hline
\multicolumn{4}{|l|}{\bgrey \textbf{Azpititulu formatuak}}\\
\hline
\grey SSA & \green Bai(natiboa) & \green Bai & \green Bai\\
\hline
\grey ASS & \red Ez & \green Bai(natiboa) & \green Bai(natiboa)\\
\hline
\grey SRT & \red Ez & \green Bai & \green Bai\\
\hline
\grey Testu planoa & \green Bai & \green Bai & \green Bai\\
\hline
\multicolumn{4}{|l|}{\bgrey \textbf{Karaktere kodifikazioa}}\\
\hline
\grey Lokala & \green Bai(natiboa) & \green Bai(natiboa) & \green Bai\\
\hline
\grey Unicode & \red Ez & \red Ez & \green Bai(natiboa)\\
\hline
\grey Autodetekzioa & \red Ez & \red Ez & \green Bai\\
\hline
\multicolumn{4}{|l|}{\bgrey \textbf{Sistema eragileak}}\\
\hline
\grey Microsoft Windows & \green Bai & \green Bai & \green Bai\\
\hline
\grey Linux & \red Ez & \red Ez & \yellow Osatugabea\\
\hline
\grey Mac OS X & \red Ez & \red Ez & \red Erabilezina\\
\hline
\multicolumn{4}{|l|}{\bgrey \textbf{Audioa}}\\
\hline
\grey PCM WAV & \yellow 8 bit mono & \green Bai & \green Bai\\
\hline
\grey MP3 & \red Ez & \green Bai & \green Bai\\
\hline
\grey Ogg Vorbis & \red Ez & \green Bai & \green Bai\\
\hline
\grey AAC & \red Ez & \red Ez & \green Bai\\
\hline
\grey AC3 & \red Ez & \red Ez & \green Bai\\
\hline
\grey Bideotik & \green Bai & \red Ez & \green Bai\\
\hline
\grey Uhin-forma & \green Bai & \green Bai & \green Bai\\
\hline
\grey Espektroa & \red Ez & \red Ez & \green Bai\\
\hline
\multicolumn{4}{|l|}{\bgrey \textbf{\textit{Typesetting}}}\\
\hline
\grey Bisualki & \red Ez & \red Ez & \green Bai\\
\hline
\grey Estilo editorea & \green Bai & \green Bai & \green Bai\\
\hline
\grey Estilo kudeatzailea & \green Bai & \green Bai & \green Bai\\
\hline
\grey Estilo aurrebista & \green Bai & \red Ez & \green Bai\\
\hline
\multicolumn{4}{|l|}{\bgrey \textbf{Audio sinkronizazioa}}\\
\hline
\grey Elkarrizketak & \green Bai & \green Bai & \green Bai\\
\hline
\grey Karaokeak & \yellow Sinplea & \green Bai & \green Bai\\
\hline
\multicolumn{4}{|l|}{\bgrey \textbf{Azpitituluen gaineko eragiketak}}\\
\hline
\grey Shift, Split eta Join & \green Bai & \green Bai & \green Bai\\
\hline
\grey Bikoiztu & \red Ez & \red Ez & \green Bai\\
\hline
\grey Undo/Redo & \red Ez & \red Ez & \green Bai\\
\hline
\grey Sintaxi nabarmendua & \red Ez & \green Bai & \green Bai\\
\hline
\grey Regex & \red Ez & \red Ez & \green Bai\\
\hline
\multicolumn{4}{|l|}{\bgrey \textbf{Tresnak}}\\
\hline
\grey Itzulpen laguntzailea & \red Ez & \red Ez & \green Bai\\
\hline
\grey Zuzentzailea & \green Bai & \red Ez & \green Bai\\
\hline
\grey Karaoke efektuak & \red Ez & \yellow Sinplea & \green Bai\\
\hline
\multicolumn{4}{|l|}{\bgrey \textbf{Bestelakoak}}\\
\hline
\grey Automatikoki gorde & \green Bai & \red Ez & \green Bai\\
\hline
\grey Segurtasun kopiak & \red Ez & \red Ez & \green Bai\\
\hline
\grey Dokumentu anitz & \red Ez & \red Ez & \red Ez\\
\end{longtable}
\section{Ondorioak}